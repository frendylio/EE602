\documentclass[11pt]{article}
\usepackage{scribe}
\usepackage{graphicx}

% Uncomment the appropriate line
%\Scribe{Your name}

\Scribes{Frendy Lio Can}
\LectureDate{November 28, 2020}
\LectureTitle{Algorithms Assignment 11}

%\usepackage[mathcal]{euscript}


\begin{document}
\MakeScribeTop

%\paragraph{This is a paragraph heading} Paragraph.

%%%%%%%%%%%%%%%%%%%%%%%%%%%%%%%%
% PROBLEM 1
%%%%%%%%%%%%%%%%%%%%%%%%%%%%%%%%
\paragraph{\noindent\textbf{\LARGE{Problem 1}}}

% Start of Explaining

\begin{flushleft}
    To prove the initial condition, we do the following: $n = 0 \Rightarrow T(0)=2^0 = 1$.
    \newline
    For $n > 0, T(n) = 1 + \sum_{j = 0}^{n - 1}T(j) = 1 + \sum_{j = 0}^{n - 1}2^j = 1 + (2^n - 1) = 2^n$
\end{flushleft}   

%%%%%%%%%%%%%%%%%%%%%%%%%%%%%%%%
% PROBLEM 2
%%%%%%%%%%%%%%%%%%%%%%%%%%%%%%%%
\paragraph{\noindent\textbf{\LARGE{Problem 2}}}

\begin{flushleft}
Algorithm(p,n,c) \newline \null
    \quad let r[n] be a new array with size n \newline \null
    \quad r[0] = 0 \newline \null
    \quad for j = 1 to n \newline \null
        \quad \quad q = p[j] \newline \null
        \quad \quad for i = 1 to j - 1 \newline \null
            \quad \quad \quad q = max(q, p[i] + r[j - i] - c) \newline \null
        \quad \quad r[j] = q \newline \null
    \quad return r[n]
\end{flushleft} 

%%%%%%%%%%%%%%%%%%%%%%%%%%%%%%%%
% PROBLEM 3
%%%%%%%%%%%%%%%%%%%%%%%%%%%%%%%%
\paragraph{\noindent\textbf{\LARGE{Problem 3}}}

\begin{flushleft}
    Let the vertices of the subproblem graph be the ordered pairs $v_{ij}$ such that $i \leq j$.
    If $i = j$ ,then there are no edges out of $v_{ij}$. If $i < j$, then for every k such that $i \leq k \leq j$,
    the subproblem graph contain edges $(v_{ij}, v_{ik})$ and $(v_{ij}, v_{i+1,j})$.
    \newline
    \newline
    Therefore, in order to solve the subproblem of optimally parenthesizing the product of $A_i$ to $A_j$, we will need to solve
    subproblem of optimal parenthesizing the products $A_i$ to $A_k$ and $A_{k+1}$ to $A_j$.
    \newline
    \newline
    Number of vertices: $\sum_{i=1}^n \sum_{j=1}^n 1= \frac{n(n+1)}{2}$
    \newpage
    Number of edges: 
\end{flushleft}
\begin{equation*}
\begin {split}
    \sum_{i=1}^n \sum_{j=1}^n (j - i) =& \sum_{i = 1}^n \sum_{a = 0}^{n-i}t \text{\quad , a = j - i}\\
    =& \sum_{i = 1}^n \frac{(n-i)(n-i+1)}{2}\\
    =& \frac{1}{2}\sum_{i = 1}^n (n-i)(n-i+1)\\
    =& \frac{1}{2}\sum_{r = 0}^{n-1} (r^2 + r) \text{\quad ,r = n - i}\\
    =& \frac{1}{2}(\frac{(n-1)n(2n-1)}{6} + \frac{(n-1)n}{2}) \\
    =& \frac{(n-1)n(n+1)}{6}
\end {split}
\end{equation*}
\begin{flushleft}
    Edges: $(v_{ij}, v_{ik})$ and $(v_{ij}, v_{i+1,j})$
\end{flushleft}

%%%%%%%%%%%%%%%%%%%%%%%%%%%%%%%%
% PROBLEM 4
%%%%%%%%%%%%%%%%%%%%%%%%%%%%%%%%
\paragraph{\noindent\textbf{\LARGE{Problem 4}}}

\begin{flushleft}
    Let $a$ be an activity array. \newline
    Let $c$ be an array that storages the sizes. \newline
    Let $s$ be an array of the start time of the activities. \newline
    Let $f$ be an array of the finish time of the activities. \newline
    Let $n$ be the size of the original problem \newline \newline
DynamicActivitySelection(s,f,n) \newline \null
    \quad let $c$ and $a$ be a 2d array of size $(n+1) x (n+1)$ \newline \null
    \quad for i in [0 ... n] \newline \null
        \quad \quad c[i,i] = 0 \newline \null
        \quad \quad c[i,i+1] = 0 \newline \null
    \quad c[n+1, n+1]= 0 \newline \null
    \quad for l in [2 ... n + 1] \newline \null
        \quad \quad for i in [0 ... n - l + 1] \newline \null
            \quad \quad \quad j = i + 1 \newline \null
            \quad \quad \quad c[i,j] = 0 \newline \null
            \quad \quad \quad k = j - 1 \newline \null
            \quad \quad \quad while f[i] < f[k] \newline \null
                \quad \quad \quad \quad  if (
                    f[i] $\leq$ s[k] AND 
                    f[k] $\leq$ s[j] AND
                    c[i,k] + c[k,j] + 1 > c[i.j]
                    ) \newline \null
                    \quad \quad \quad \quad \quad c[i,j] = c[i,k] + c[k,j] + 1 \newline \null
                    \quad \quad \quad \quad \quad a[i,j] = k \newline \null
                \quad \quad \quad \quad k = k - 1 \newline \null
    \quad print "The maximum size set of mutually compatible activities has size:" c[0, n+1] \newline \null
    \quad print "The set contains:" PrintActivities(c, a, i, j) \newline \null
PrintActivities(c, a, i, j) \newline \null
    \quad if[i, j] > 0 \newline \null
        \quad \quad k = a[i, j] \newline \null
        \quad \quad print k \newline \null
        \quad \quad PrintActivities(c, a, i, kj) \newline \null
        \quad \quad PrintActivities(c, a, k, j) \newline \null

\end{flushleft}

%%%%%%%%%%%%%%%%%%%%%%%%%%%%%%%%
% PROBLEM 5
%%%%%%%%%%%%%%%%%%%%%%%%%%%%%%%%
\paragraph{\noindent\textbf{\LARGE{Problem 5}}}

\begin{flushleft}
If we choose the last activity to start  that is compatible with all previously selected activities; the algorithm is similar to selecting the first activity to finish.
The only difference it that when we do last activity to start, the algorithm will be running in reverse. Thus, since 
they are similar, it will produce an optimal solution. 
\newline 
\newline
This approach is a greedy algorithm because at each step we make the best choice.
\end{flushleft}    

%%%%%%%%%%%%%%%%%%%%%%%%%%%%%%%%
% PROBLEM 6
%%%%%%%%%%%%%%%%%%%%%%%%%%%%%%%%
\paragraph{\noindent\textbf{\LARGE{Problem 6}}}
\begin{flushleft}

\end{flushleft}    
    

%%%%%%%%%%%%%%%%%%%%%%%%%%%%%%%%
% PROBLEM 7
%%%%%%%%%%%%%%%%%%%%%%%%%%%%%%%%
\paragraph{\noindent\textbf{\LARGE{Problem 7}}}
\begin{flushleft}
    For every stack operation, we add to our cost the actual cost of a stack operation and the cost of copying an element.
    This implies that for every stack operation, we add 2 times to the cost.
    \newline
    \newline
    We also know that the size of stack will not exceed k, and there are always k operations between backups; 
    we will overpay by at least enough cost. 
    Therefore, the amortized cost of the operation is constant, and the cost of the n operation is $O(2n) = O(n)$
\end{flushleft}

%%%%%%%%%%%%%%%%%%%%%%%%%%%%%%%%
% PROBLEM 8
%%%%%%%%%%%%%%%%%%%%%%%%%%%%%%%%
\paragraph{\noindent\textbf{\LARGE{Problem 8}}}
\begin{flushleft}
    Using the INCREMENT Algorithm from page (454) from CLRS.
    \newline
    \newline
    Let high be a global variable that is initialize at $-\infty$.
    \newline
    \newline
    INCREMENT(A) \newline \null
        \quad i = 0 \newline \null
        \quad while i < A.length and A[i] == 1 \newline \null
            \quad \quad A[i] = 0 \newline \null
            \quad \quad i = i + 1 \newline \null
        \quad if i < A.length \newline \null
            \quad \quad A[i] = 1 \newline \null
            \quad \quad if i > high \newline \null
                \quad \quad \quad high = 1
    \newline
    \newline
    RESET(A) \newline \null
        \quad while high > 0 \newline \null
            \quad \quad A[high] = 0 \newline \null
            \quad \quad high = high - 1 \newline \null
        \quad A[high] = 0
    \newline
    \newline
    We can observe that INCREMENT(A) and RESET(A) operations will take $O(1)$ time by the accounting method.
    This is because when a bit is set to 1, we carry a cost of 1. This cost or bit can be used in the RESET operation as well which will take $O(1)$ time.
    \newline
    \newline
    Therefore, the amortized cost is $O(n)$ which implies that the total cost is also $O(n)$.
\end{flushleft}
%%%%%%%%%%%%%%%%%%%%%%%%%%%%%%%%
% PROBLEM 9
%%%%%%%%%%%%%%%%%%%%%%%%%%%%%%%%
\paragraph{\noindent\textbf{\LARGE{Problem 9}}}
\begin{flushleft}
    If $\phi(D_i)  = \sum_{k = 1}^i lg k$, this implies that the amortized cost for INSERT is 
    $\hat{c_i} = c_i + \phi(D_i) - \phi(D_i - 1) \leq lg i + \sum_{k=1}^i lg k - \sum_{k=1}^{i-1}lg k = 2 lg i \rightarrow O(lgn)$.
    \newline
    \newline
    The amortized cost for EXTRACT-MIN will be 
    $\hat{c_i} = c_i + \phi(D_i) - \phi (D_i -1) \leq lg 1 + 1 + \sum_{k=1}^{i-1} lg k - \sum_{k=1}^i lg k = 1 \rightarrow O(1)$
\end{flushleft}


\end{document}


