\documentclass[11pt]{article}
\usepackage{scribe}
\usepackage{graphicx}

% Uncomment the appropriate line
%\Scribe{Your name}

\Scribes{Frendy Lio Can}
\LectureDate{September 23, 2020}
\LectureTitle{Algorithms Assignment 4}

%\usepackage[mathcal]{euscript}


\begin{document}

\MakeScribeTop

%\paragraph{This is a paragraph heading} Paragraph.

%%%%%%%%%%%%%%%%%%%%%%%%%%%%%%%%
% PROBLEM 1
%%%%%%%%%%%%%%%%%%%%%%%%%%%%%%%%
\paragraph{\noindent\textbf{\LARGE{Problem 1}}}

% Start of Explaining

\begin{flushleft}
    To prove this, we have to show that there exists constants $c_1, c_2, n_0 > 0$ such that 
    $0 \leq c_1 n^b \leq (n + a)^b \leq c_2 n^b$ , $\forall n \geq n_0 $. 
    \newline \newline
    We also can observe that $n + a \leq 2n$ , when $|a| \leq n$; and $n + a \geq \frac{1}{2}n$ , when $|a| \leq \frac{n}{2}$.
    \newline \newline
    Thus, if $n \geq 2|a|$
\end{flushleft}   
\begin{equation*}
\begin{split}
     & 0 \leq \frac{n}{2} \leq n + a \leq 2n \Leftrightarrow \\
     \Leftrightarrow \quad & 0^b \leq \frac{n}{2}^b \leq (n + a)^b \leq (2n)^b , \quad b > 0 \\
    \Leftrightarrow \quad & 0 \leq 2^{-b} n^b \leq (n + a)^n \leq 2^b n^b
\end{split}
\end{equation*}
\begin{flushleft}
    $\therefore{} \quad (n+a)^b = \Theta(n^b) \quad \because \quad \exists c_1, c_2, n_0 : c_1 = 2^{-b}, c_2 = 2^b, n_0 = 2|a|$ \quad when $b > 0$
\end{flushleft}   

%%%%%%%%%%%%%%%%%%%%%%%%%%%%%%%%
% PROBLEM 2
%%%%%%%%%%%%%%%%%%%%%%%%%%%%%%%%
\paragraph{\noindent\textbf{\LARGE{Problem 2}}}

\begin{flushleft}
    Yes, $2^{n+1} =  O(2^n)$ and $2^2n \neq O(2^n).$ \newline\newline
    For $2^{n+1} =  O(2^n)$
    \newline

    Assume that $2^{n+1} =  O(2^n)$. Thus, we need to prove that there exists constants $c, n_0 > 0 $ such that $0 \leq 2^{n+1} \leq c 2^n \quad ,\forall n \geq n_0.$
    We can observe that 
    $0 \leq 2^{n+1} \leq c 2^n \Leftrightarrow 0 \leq 2 \cdot 2^n \leq c 2^n$.
    \newline \newline
    Clearly, we can observe that there exists constants $c, n_0 > 0$ such that $c_0 \geq 2, n \geq 1$. Therefore $2^{n+1} =  O(2^n)$.
    \newline\newline
    For $2^2n \neq O(2^n)$
    \newline

    Assume that $2^2n = O(2^n)$. Thus, we need to prove that there exists constants $c, n_0 > 0 $ such that $0 \leq 2^{2n} \leq c 2^n \quad ,\forall n \geq n_0.$
    We can observe that 
    $0 \leq 2^{2n} \leq c 2^n \Leftrightarrow 0 \leq 2^n \cdot 2^n \leq c 2^n$.
    \newline \newline
    Clearly, we can observe that there does not exists any constants $c$ when $n \rightarrow \infty$. This contradicts the assumption $2^2n = O(2^n)$. Therefore, we can conclude that $2^2n \neq O(2^n)$
    \newline
\end{flushleft}

%%%%%%%%%%%%%%%%%%%%%%%%%%%%%%%%
% PROBLEM 3
%%%%%%%%%%%%%%%%%%%%%%%%%%%%%%%%
\paragraph{\noindent\textbf{\LARGE{Problem 3}}}

\begin{flushleft}
    If $T(n) = O(n^2)$, there must exists constants $c, n_0 > 0$ such that $ T(n) \leq cn^2, \quad \forall   n \geq n_0$.
\end{flushleft}
\begin{equation*}
\begin {split}
    T(n) = T(n-1) + n &\leq c(n-1)^2 + n    \\
    &\leq cn^2 - 2cn + 1 + n \\
    &\leq cn^2 + (-2c + 1)n + 1 \\
    &\leq cn^2 + (-2c + 1)n \\
    &\leq cn^2 - 2cn \\
    &\leq cn^2
\end {split}
\end{equation*}
\begin{flushleft}
    We can observe that for $n \rightarrow \infty$ and $c \geq 1$, \quad $T(n) \leq cn^2$. Therefore, $T(n) = O(n^2)$ .
\end{flushleft}

%%%%%%%%%%%%%%%%%%%%%%%%%%%%%%%%
% PROBLEM 4
%%%%%%%%%%%%%%%%%%%%%%%%%%%%%%%%
\paragraph{\noindent\textbf{\LARGE{Problem 4}}}

\begin{flushleft}
    Recursion tree:    
\end{flushleft}

\Tree [.$cn$
        [.$c\alpha n$ 
          [.$c\alpha^2n$ 
              [.$\vdots$ ] 
          ] 
          [.$c(1-\alpha)\alpha n$ 
              [.$\vdots$ ] 
          ] 
        ] 
        [.$c(1-\alpha)n$
          [.$c(1-\alpha)\alpha n$ 
              [.$\vdots$ ] 
          ] 
          [.$c(1-\alpha)^2n$
              [.$\vdots$ ] 
          ] 
        ] 
      ] 
\Tree 
[.$cn$
  [.$cn$
    [.$cn$
      [.$\vdots$ ] 
    ]  
  ] 
]
\begin{flushleft}
    We can observed that $f(n) = cn$, $a = c$ and $b = 1/(1- \alpha)$.
    \newline
    
    Therefore, in order to calculate the lower bound, we add the cost of each level of the tree by knowing that the tree is completed when the Recursion tree level is $log_b n$:
\end{flushleft}
\begin{equation*}
\begin {split}
    T(n) =& \sum_{j = 0}^{log_b \cdot n} f(n) \\
         =& \sum_{j = 0}^{log_b \cdot n} cn \\
         =& \sum_{j = 1}^{log_b \cdot n} cn + cn \\
         \geq & n log_b n \\
         \geq & n lg n \\
    \therefore{} T(n) = &\Omega(n lg n)
\end {split}
\end{equation*}
\begin{flushleft}
    In order to calculate the upper bound, we assume that there exists a constant b such that $T(n) \leq b \cdot n \cdot lgn$.
    \newline

    Thus, the upper bound is:
\end{flushleft}
\begin{equation*}
\begin {split}
    T(n) =& T(\alpha n) + T((1 - \alpha)n) + cn \\
        \leq & b (\alpha  n) \cdot lg(\alpha n) + b[(1-\alpha)n] \cdot lg[(1 - \alpha)n] + cn \\
        \leq & [b\alpha n lg \alpha + b\alpha n lg n] + [b(1-\alpha)n lg(1 - \alpha) + b(1 - \alpha) n lg n] + cn \\
        \leq & bnlgn + bn(\alpha lg \alpha + (1- \alpha) lg (1-\alpha)) + cn \\
        \leq & bnlgn 
\end {split}
\end{equation*}
\begin{flushleft}
    We can apply the same logic and find the lower bound which is:
\end{flushleft}
\begin{equation*}
\begin {split}
    T(n)    \geq & bnlgn 
\end {split}
\end{equation*}
\begin{flushleft}
    We can conclude that $T(n)=\Theta(nlgn)$, for any $n \geq n_0$ and $c \geq -b(\alpha lg \alpha + (1- \alpha) lg(1 - \alpha))$.
\end{flushleft}
%%%%%%%%%%%%%%%%%%%%%%%%%%%%%%%%
% PROBLEM 5
%%%%%%%%%%%%%%%%%%%%%%%%%%%%%%%%
\paragraph{\noindent\textbf{\LARGE{Problem 5}}}

\begin{flushleft}
    % =============================
   a) $T(n) = 2T(\frac{n}{2}) + n^3$
    \newline

   We have $a = 2, b = 2, f(n) = n^3$ and $n^{log_ba} = n^{log_22}= n$. 
   We can observe that case 3 of the master method applies since $f(n) = \Omega(n^{log_ba+\epsilon}), \epsilon = 3$. 
    \newline

   Therefore, $T(n) = \Theta(n^3)$.
    \newline

 % =============================
    b) $T(n) = T(\frac{9n}{10}) + n$
    \newline

   We have $a = 1, b = 10/9, f(n) = n$ and $n^{log_{10/9}1} = n^0= 1$. 
   We can observe that case 3 of the master method applies since $f(n) = \Omega(n^{log_ba+\epsilon}), \epsilon = 1$. 
    \newline

   Therefore, $T(n) = \Theta(n)$.    
   \newline
 % =============================

    c) $T(n) = 16T(\frac{n}{4}) + n^2$
    \newline

   We have $a = 16, b = 4, f(n) = n^2$ and $n^{log_{4}16} = n^2$. 
   We can observe that case 2 of the master method applies since $f(n) = \Theta(n^{log_ba})$. 
    \newline

   Therefore, $T(n) = \Theta(n^{log_ba}lgn) = \Theta(n^{2}lgn)$.    
   \newline
 % =============================

 d) $T(n) = 7T(\frac{n}{3}) + n^2$
 \newline

We have $a = 7, b = 3, f(n) = n^2$ and $n^{log_{3}7} = n^{1.77}$. 
We can observe that case 3 of the master method applies since $f(n) = \Omega(n^{log_ba+\epsilon}), \epsilon \approx 0.23$. 
 \newline

Therefore, $T(n) = \Theta(n^2)$.    
\newline
 % =============================

 e) $T(n) = 7T(\frac{n}{2}) + n^2$
 \newline

We have $a = 7, b = 2, f(n) = n^2$ and $n^{log_{2}7} = n^{2.81}$. 
We can observe that case 1 of the master method applies since $f(n) = \Omega(n^{log_ba-\epsilon}), \epsilon \approx 0.81$. 
 \newline

Therefore, $T(n) = \Theta(n^{log_ba}) = \Theta(n^{log_27})$.    
\newline
 % =============================

 f) $T(n) = 2T(\frac{n}{4}) + n^{1/2}$
 \newline

We have $a = 2, b = 4, f(n) = n^{1/2}$ and $n^{log_{4}2} = n^{1/2}$. 
We can observe that case 2 of the master method applies since $f(n) = \Omega(n^{log_ba})$. 
 \newline

Therefore, $T(n) = \Theta(n^{log_42}lgn)$.    
\newline
 % =============================

 g) $T(n) = T(n - 1) + n$
\end{flushleft}
\begin{equation*}
    \begin{split}
    T(n) = &T(n - 1) + n \\
    T(n-1) = &T(n - 2) + n - 1 \\
    T(n-2) = &T(n - 3) + n - 2
\end{split}
\end{equation*}
Therefore,
\begin{equation*}
\begin{split}
    T(n) = &T(n - 1) + n = T(n - 2) + n - 1 + n = T (n - 3) + n - 2 + n - 1 + n \\
    T(n) = &T(n - k) + kn - k(k-1)/2
\end{split}
\end{equation*}
Thus, if our base case for $T(1)$ is $n-k = 1 \Rightarrow k = n - 1$. Thus, 
\begin{equation*}
\begin{split}
    T(n) = &T(1) + (n - 1)n - \frac{(n - 1)(n -2)}{2}
\end{split}
\end{equation*}
We can clearly observed that it has complexity $\Theta(n^2)$.

\begin{flushleft}
 % =============================

 h) $T(n) = T(n^{1/2}) + 1$
 \newline

 Let $n = 2^m$. Thus, $T(2^m) = T(2^{m \cdot 1/2}) + 1$. Assume tht $T(2^m) = S(m)$ where S is some function of m.
 \newline

 Thus, $S(m) = S(m/2) + 1$
 \newline

We have $a = 1, b = 2, f(n) = 1$ and $n^{log_{2}1} = 1$. 
We can observe that case 2 of the master method applies since $f(n) = \Omega(1)$. 
 \newline

Therefore, $S(m) = \Theta(lgm)$. Thus, since $n = 2^m \Rightarrow lg n = m$, $T(n) = \Theta(lg (lg n))$    
\newline

\end{flushleft}

\end{document}
